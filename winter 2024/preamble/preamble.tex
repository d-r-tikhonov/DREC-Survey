% ==================================================

\usepackage[margin = 1in]{geometry}
\usepackage[utf8]{inputenc}
\usepackage[T1, T2A]{fontenc}
\usepackage{mathtext}
\usepackage[english, russian]{babel}
\usepackage{indentfirst}
\frenchspacing

% ==================================================

% Packages for math
\usepackage{mathrsfs}
\usepackage{amsfonts}
\usepackage{amsmath}
\usepackage{amsthm}
\usepackage{amssymb}
\usepackage{physics}
\usepackage{dsfont}
\usepackage{esint}

% ==================================================

% Packages for writing
\usepackage{enumerate}
\usepackage[shortlabels]{enumitem}
\usepackage{framed}
\usepackage{csquotes}

% ==================================================

% Miscellaneous packages
\usepackage{float}
\usepackage{tabularx}
\usepackage{xcolor}
\usepackage{multicol}
\usepackage{subcaption}
\usepackage{caption}
\captionsetup{format = hang, margin = 10pt, font = small, labelfont = bf}
\renewcommand{\thefigure}{\hspace{-.333333em}}

\usepackage[bottom]{footmisc}
\renewcommand{\thefootnote}{*}

%%% Колонтитулы
\usepackage[pagestyles]{titlesec}
\newpagestyle{main}{
	\setheadrule{0.4pt}
	\sethead{Результаты проведения опроса о качестве преподавания}{}{\hyperlink{intro}{\;Назад к содержанию}}
	\setfootrule{0.4pt}
	\setfoot{ФРКТ МФТИ, 2025}{}{\thepage} 
}
\pagestyle{main}

%%% Оформление страницы
\usepackage{extsizes}     % Возможность сделать 14-й шрифт
\usepackage{geometry}     % Простой способ задавать поля
\usepackage{setspace}     % Интерлиньяж

\geometry{top=25mm}    % Поля сверху страницы
\geometry{bottom=30mm} % Поля снизу страницы
\geometry{left=20mm}   % Поля слева страницы
\geometry{right=20mm}  % Поля справа страницы

\setlength\parindent{15pt}        % Устанавливает длину красной строки 15pt
\linespread{1.3}                  % Коэффициент межстрочного интервала
%\setlength{\parskip}{0.5em}      % Вертикальный интервал между абзацами
%\setcounter{secnumdepth}{0}      % Отключение нумерации разделов
%\setcounter{section}{-1}         % Нумерация секций с нуля
\usepackage{multicol}			  % Для текста в нескольких колонках
\usepackage{soulutf8}             % Модификаторы начертания

% Hyperlinks setup
\usepackage{hyperref}
\definecolor{links}{rgb}{0.36,0.54,0.66}
\hypersetup{
    colorlinks  = true,
    linkcolor   = black,
    urlcolor    = blue,
    citecolor   = blue,
    filecolor   = blue,
    pdfauthor   = {Author},
    pdftitle    = {Title},
    pdfsubject  = {subject},
    pdfkeywords = {one, two},
    pdfproducer = {LaTeX},
    pdfcreator  = {pdfLaTeX},
}

% ==================================================

\usepackage{titlesec}
\usepackage[many]{tcolorbox}

% Adjust spacing after the chapter title
\titlespacing*{\chapter}{0cm}{-2.0cm}{0.50cm}
\titlespacing*{\section}{0cm}{0.50cm}{0.25cm}

% Indent 
\setlength{\parindent}{0pt}
\setlength{\parskip}{1ex}

\newcounter{commentcounter}
\setcounter{commentcounter}{0}

% Стиль для комментариев студентов
\newtcolorbox{commentbox}{
    enhanced,
    colback=white,
    colbacktitle=white,
    coltitle=black,
    boxrule=0pt,
    frame hidden,
    borderline west={0.5mm}{0.0mm}{black},
    fonttitle=\bfseries\sffamily,
    breakable
}
